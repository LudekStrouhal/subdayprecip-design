% Version 2016-11-14
% update – 161114 by Ken Arroyo Ohori: made spacing closer to Word template throughout, put proper quotes everywhere, removed spacing that could cause labels to be wrong, added non-breaking and inter-sentence spacing where applicable, removed explicit newlines

\documentclass{isprs}
\usepackage{subfigure}
\usepackage{setspace}
\usepackage{geometry} % added 27-02-2014 Markus Englich
\usepackage{epstopdf}
\usepackage[labelsep=period]{caption}  % added 14-04-2016 Markus Englich - Recommendation by Sebastian Brocks

\geometry{a4paper, top=25mm, left=20mm, right=20mm, bottom=25mm, headsep=10mm, footskip=12mm} % added 27-02-2014 Markus Englich
%\usepackage{enumitem}

%\usepackage{isprs}
%\usepackage[perpage,para,symbol*]{footmisc}

%\renewcommand*{\thefootnote}{\fnsymbol{footnote}}
\captionsetup{justification=centering} % thanks to Niclas Borlin 05-05-2016

\usepackage{listings}
\lstset{basicstyle=\ttfamily\scriptsize,
  showstringspaces=false}

\usepackage{url}

\begin{document}

\title{Building a complete free and open source GIS infrastructure for hydrological computing and data publication using GIS.lab and Gisquick platforms}

% KAO: Remove extra spacing
\author{
  M. Landa\textsuperscript{a}, P. Kavka\textsuperscript{b}, L. Strouhal\textsuperscript{b}, J. Cepicky\textsuperscript{c}
}

% KAO: Remove extra newline
\address{
  \textsuperscript{a }Dept.\ of Geomatics, Faculty of Civil Engineering, Czech Technical University in Prague, Czech Republic - martin.landa@fsv.cvut.cz\\
  \textsuperscript{b }Dept.\ of Irrigation, Drainage and Landscape Engineering, Czech Technical University in Prague, Czech Republic - \\
  (petr.kavka, ludek.strouhal)@fsv.cvut.cz\\
  \textsuperscript{c }OpenGeoLabs s.r.o., Prague, Czech Republic - jachym.cepicky@opengeolabs.cz
}

% If the corresponding author is NOT the final author, always add a % space before the subsequent comma, i.e.
% first author name\textsuperscript{a,}\thanks{Corresponding author} , % second author name \textsuperscript{b}, etc.
% thanks to Niclas Borlin 05-05-2016


\commission{IV, }{IV} %This field is optional.
\workinggroup{IV/4} %This field is optional.
\icwg{}   %This field is optional.

% KAO: Use times symbol
\abstract{ Building a complete free and open source GIS computing and
  data publication platform can be a relatively easy task. This paper
  describes an automated deployment of such platform using two open
  source software projects -- GIS.lab and Gisquick. GIS.lab
  (\url{http://web.gislab.io}) is a project for rapid deployment of a
  complete, centrally managed and horizontally scalable GIS
  infrastructure in the local area network, data center or cloud. It
  provides a comprehensive set of free geospatial software seamlessly
  integrated into one, easy-to-use system. A platform for GIS
  computing (in our case demonstrated on hydrological data processing)
  requires core components as a geoprocessing server, map server, and
  a computation engine as eg. GRASS GIS, SAGA, or other similar GIS
  software. All these components can be rapidly, and automatically
  deployed by GIS.lab platform. In our demonstrated solution PyWPS is
  used for serving WPS processes built on the top of GRASS GIS
  computation platform. GIS.lab can be easily extended by other
  components running in Docker containers. This approach is shown on
  Gisquick seamless integration. Gisquick (\url{http://gisquick.org})
  is an open source platform for publishing geospatial data in the
  sense of rapid sharing of QGIS projects on the web. The platform
  consists of QGIS plugin, Django-based server application, QGIS
  server, and web/mobile clients. In this paper is shown how to easily
  deploy complete open source GIS infrastructure allowing all required
  operations as data preparation on desktop, data sharing, and
  geospatial computation as the service. It also includes data
  publication in the sense of OGC Web Services and importantly also as
  interactive web mapping applications.  }

\keywords{GIS, Open Source, Free Software, Deployment, Hydrology, GIS.lab, Gisquick}

\maketitle

\end{document}
